

	Cena: uma jovem de cabelos claros, envolta num véu diáfano e vestindo uma coroa de galhos retorcidos, de olhos fechados, com as mãos erguidas num gesto ritualístico, de onde emana uma leve luz dourada. Um canto coral harmonioso soa ao redor dela. Porém, por trás da jovem se esgueiram três criaturas ferais, peludas e musculosas, rosnando e babando enquanto se preparam para o ataque.
	-- Déia, GNOLLS! -- grita uma voz grossa e masculina.
	-- OK! -- responde uma voz mais fina.
	A jovem continua parada, envolvida no seu ritual, quando uma figura salta por trás dela e esmaga a cabeça do primeiro gnoll com um machado. O segundo gnoll levanta seu facão para se defender, mas a figura lança um chute no peito que o derruba no chão. O terceiro gnoll tem tempo de descer o facão em cima da perna da oponente, mas ela nem reage ao golpe, girando o machado para amassar mais uma cabeça. 
	-- YES! Dois headshots! A bárbara salva o dia de novo! -- exclama a voz mais fina, enquanto a bárbara vira para enfrentar o gnoll que se levanta do chão.
	A jovem no véu continua imersa no seu ritual.
	-- Como tá aí, Rô? -- pergunta a voz grossa. Uma terceira voz, tenor, responde.
	-- Ainda tenho dois resists e duas poções, mas ela já detonou um terço do segundo glifo. Só não pára o ritual, Luís. Sério, não pára esse ritual por nada.
	Ali perto, uma guerreira de armadura pesada com grossas tranças cor-de-rosa se esconde atrás de um escudo gigantesco. Na frente dela há uma criatura que brilha como fogo, e é grande como uma casa, com asas douradas enormes. A Fênix solta um guincho agudo que faz todos grunhirem de aflição, e solta uma baforada de fogo sobre a guerreira. O jato de fogo a engole completamente, por vários segundos tensos. Finalmente, as chamas diminuem e a guerreira reaparece, com as tranças intactas dentro de um escudo de força. A turma suspira. Rô quebra o silêncio.
	-- Ok, um resist agora. Arya, cadê você?
	-- Me dá um minuto, Rô, tô voltando aí. -- responde uma quarta voz, com um jeito infantil.
	-- Hm, beleza, eu agüento um minuto aqui de boas se Diana matar o que ela summonar agora.
	A Fênix está fazendo gestos com as garras e a cabeça, soltando guinchos curtos e balançando as asas. Depois de cinco guinchos, dezenas de monstros começam a entrar na caverna, se arrastando, dos túneis ao redor.
	-- UNDEAD! -- grita Déia, em frustração. A bárbara está atacando os mortos-vivos, derrubando dois com cada golpe, mas mal consegue mantê-los longe da sacerdotisa. Alguns zumbis mordem as pernas da guerreira de armadura, e ela tenta espantá-los com chutes enquanto segura o escudo enorme.

-- PQP, Rô, você tinha que falar. -- resmunga Luís, -- Vou ter que quebrar o ritual, bosta.

-- Bom, faz isso agora. Ela começa a atacar em seis segundos.

-- Já fiz, mas não vai dar tempo.

Esqueletos animados brotam do chão da caverna e começam a atacar as guerreiras. A bárbara destrói três deles com um só golpe, mas dois passam em direção à sacerdotisa. Então, a sacerdotisa abre os olhos, e o brilho e o coral desaparecem. Ela proclama um encantamento, e uma explosão de luz se espalha em todas as direções. Imediatamente, todos os mortos vivos atingidos se desfazem em pó.

-- Aí sim! -- exclama Andreanna com tom de alívio. Agora a bárbara destrói os mortos-vivos à medida em que eles brotam do chão.

-- "Aí sim" nada, estamos sem proteção. -- resmunga Luís, suando. -- O ritual já era. Estou só curando vocês.

A Fênix pára subitamente os guinchos, e tenta saltar por sobre a guerreira de armadura, mas é interrompida por uma escudada no bico. Imediatamente ela ataca o escudo com uma bicada violenta. O escudo solta intensas luzes verdes, e os dois sinais brilhantes inscritos nele se apagam. Rô xinga e bate com força no teclado. A guerreira faz uma pose defensiva e um campo de força a envolve, defendendo-a de mais bicadas violentas do pássaro. Rô xinga.

-- Dois glifos numa só! E acabaram os resists guardados. Os próximos ataques especiais vão me matar com certeza.

-- Eu disse que a gente não tinha nível suficiente pra essa quest. -- resmunga Luís, resignado. -- Você pode pelo menos cobrir a nossa fuga?

-- Cara, sem você eu morro em dez segundos.

-- Vamos todos morrer, então? Puxa vida, eu cuidei tão bem dessa personagem... Olha isso, Liz, isso é o que você ganha por ficar se associando com essas baderneiras...

Luís conversa com sua personagem. A guerreira de armadura resiste às bicadas da fênix. A bárbara mata os últimos monstros da caverna. Déia se endireita na sua cadeira, e olha para os companheiros.

-- Não dá pra interromper os ataques especiais como a gente fez quando entrou?

Rô volta sua atenção para ela.

-- Dá se a gente bater nela, mas sem o especial ela começa a baforada de fogo mais rápido, lembra? -- ele fecha os olhos, frustrado. -- Rosa morre de qualquer jeito.

-- Bom, qualquer coisa pra vocês pararem de choramingar. Vou puxar o aggro pra você desfazer o resist. -- ela sorri e olha para a Fênix. 
-- Vamos começar a dar dano nesse bicho.

-- Hm, se eu desfizer agora eu ganho bônus de cool-down... Luís pode me dar outro bônus... Vai ficar em cima. E se der errado você cobre nossa fuga?

-- Se der errado a gente faz novos persongens.

Há um instante de silêncio e uma decisão é tomada. Diana, a bárbara, salta sobre a fênix com seu machado, e a fênix contra-ataca com garras e bicadas. A barra de vida de bárbara diminui numa velocidade espantosa. Rosa, a guerreira de armadura, desfaz sua pose defensiva e começa a atacar a fênix com um martelo de batalha. Liz, a sacerdotisa, recita encantamentos que fazem nascer um tapete de folhas em direção à guerreira. A contagem regressiva da habilidade de Resistir de Rosa marca 20 segundos, mas começa a descer mais rápido.

-- Agora, as asas! -- grita Rô.

A Fênix abre as asas e começa a se levantar como que para alçar vôo. 
Rosa pula sobre a asa direita com seu martelo, e Diana dá uma machadada na asa esquerda.
A ave guincha, sacode as asas, afastando as duas guerreiras, cambaleia um pouco, e recomeça a abrir as asas.

-- Outra vez!

Diana salta sobre a asa novamente, e finca nela seu machado. A Fênix grita, mas continua se levantando.
Rosa corre para martelar a asa direita, mas golpeia o vazio.
A Fênix tenta bater as asas, e derruba a bárbara. Ela se levanta completamente e se prepara para saltar para o ataque especial.
De repente, a sacerdotisa arremessa uma bola de folhas na asa direita. A bola estoura no contato,
e a asa se cobre de ramos verdes. A Fênix pára seu movimento, incapaz de abrir completamente a asa, 
solta um novo guincho e volta a atacar a bárbara com bicadas. Oito segundos.
A bárbara está perto da morte, mas Rosa a defende com seu escudo, batendo no bico da Fênix. Seis segundos.
A Fênix solta o guincho agudo que anuncia o jato de fogo. A contagem regressiva marca três segundos.

Não vai dar tempo.

-- Fuuuuuuuuuuuu!! -- Rô martela continuamente a tecla que vai liberar o campo de força, com mêdo de perder um décimo de segundo.
A Fênix abre o bico. A bárbara se esconde atrás do escudo da guerreira. Dois segundos. 
A baforada de fogo atinge o escudo e envolve as duas personagens, ocultando até mesmo as barras de vida.
A contagem regressiva acabou, e Rô pára de bater na tecla.

-- Hm. Ari, talvez ainda dê tempo de você sair dessa e ficar com o loot?

-- Ou pode ter dado tempo de eu salvar a pele de vocês. -- responde Ariadne, voltando para a conversa depois de minutos jogando sozinha.

A baforada de fogo acaba, e as duas guerreiras reaparecem, dentro do escudo de força, com os cabelos e as roupas chamuscadas, mas vivas. Aparece na cena uma nova personagem, uma pessoa pequena num traje inteiro preto, carregando um saco enorme.

-- YES! -- grita Rô. -- Sra Fênix, valou, faleu! Ari, o que você fez?

-- Resistência a fogo, redução de ataque, e lentidão. Achei que você ia perceber essa última pelo menos, pô. E agora... névoa pra cobrir nossa fuga?

-- Isso.

As quatro personagens já correm para um túnel de saída, enquanto a Fênix guincha e novos monstros surgem na caverna. Uma névoa as envolve, e os monstros ficam para trás. Rosa fecha uma porta pesada atrás de si.

Silêncio.

-- Nós conseguimos.

-- Pausa para uma pizza?

